\documentclass[../main.tex]{subfiles}

\begin{document}

\begin{enumerate}
    \item Download the source code\\
    
    \item Navigate to the source code root directory\\
    This is the directory where you can see \textbf{adjuster/}, \textbf{core/}, and \textbf{rest/} directories.

    \item Compile the application\\
    Run \textbf{mvn clean install -DskipTests}. Do not forget to link your Maven to Java 21, otherwise the build will not be successful.

    \item Navigate to \textbf{rest/} directory\\

    \item Run the application in Quarkus dev mode\footnote{\url{https://quarkus.io/guides/getting-started##development-mode}}\\
    Run \textbf{quarkus dev}. Reqour uses Quarkus version \textit{3.16.3}\footnote{\url{https://github.com/project-ncl/reqour/blob/akridl-thesis/pom.xml##L82}}, so in case of problems, explicitly specify this version.

    \item Start requesting the running application\\
    You can request the application directly from any HTTP client, e.g. \textit{curl}. Because Reqour exposes its API using OpenAPI thanks to \textit{quarkus-smallrye-openapi} extension\footnote{\url{https://quarkus.io/guides/openapi-swaggerui}}, you can open Swagger UI at the default URL: \url{http://localhost:8080/q/swagger-ui/}.

    \textbf{WARNING:} Be aware that requests working with external systems (namely: \textit{POST /adjust} and \textit{POST /cancel} — both working with OpenShift, and \textit{POST /internal-scm} — working with GitLab) will \textbf{NOT} work out of the box. In both cases are needed own resources (an OpenShift cluster in first 2 cases, and a GitLab workspace in the third case) and corresponding secrets to these resources (OpenShift and GitLab API Group tokens) which are not provided publicly from security reasons.
\end{enumerate}

\end{document}