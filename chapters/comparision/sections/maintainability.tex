\documentclass[../main.tex]{subfiles}

\begin{document}

Maintainability can be defined as: "The ease with which a software system or component can be modified to correct faults, improve performance
or other attributes, or adapt to a changed environment"\cite{software-glossary}. Many factors effect maintainability of software system. Let us list several of them and compare Repour and Reqour against them:
\begin{itemize}
    \item \textbf{Understanding of used programming language / framework}\\
    Repour is written in Python. On the one hand, Python is a high-level language with many programmers knowing the basics of it, since it is often taught as one of the first languages, e.g. at universities. On the other hand, having advanced Python knowledge, e.g. understanding internals and correct usage of asyncio library, is not common among programmers and requires non-trivial amount of time for being confident with it. Other PNC microservices are written either in Quarkus or plain Jakarta EE. Quite naturally, this lead to a situation when many programmers focused on Quarkus and did not want to invest that much time into learning Repour. The result was that there was a single person with deep understanding of Repour, taking care of its maintenance. By rewriting Repour into Reqour (written in Quarkus), the bus factor (regarding Repour - Reqour maintenance) increased from 1 to all PNC backend developers.

    \item \textbf{Documentation}\\
    Both Repour and Reqour contain extensive documentation (with only most trivial parts not being documented). Hence, one could say there is no benefit of Reqour against Repour at this factor. However, more often than not, long-living systems are being maintained but their documentation gets more and more outdated compared to reality in source code. This was also the case of Repour in many cases. Reqour, since being newly implemented, does not suffer from this defect.

    \item \textbf{Unused code}\\
    As the system evolves, the amount of unused code typically increases. Repour is no exception and contains an unused code. Examples worth mentioning are old system for storing SCM repositories\footnote{This was migrated for an internal GitLab.} or few outdated mechanisms to run the alignment operation.
\end{itemize}

To conclude, all factors mentioned above are putting Reqour into better position regarding maintainability. However, we would not like to blame Repour for being impossible to maintain in any way. Predominantly, these reasons come from Repour being a long-living system and suffering from typical problems of such system.

\end{document}
