\documentclass[../main.tex]{subfiles}

\begin{document}

First things first, we need to unify a terminology: A \textbf{cluster} is two or more computers working together to provide higher availability, reliability, and scalability than can be obtained by using a single system. When failure occurs in a cluster, resources are redirected and the workload is redistributed\footnote{\url{https://learn.microsoft.com/en-us/previous-versions/windows/it-pro/windows-server-2003/cc778629(v=ws.10)}}. By \textbf{clusterability} of an application, we mean an ability of an application to exist in several replicas (within the same cluster) and provide the following guarantee: during any request, data consistency is preserved.

From the above, it should be clear that stateless applications are predetermined to support clusterability effortlessly (since they do not store any state), unlike stateful applications. As a consequence, stateless applications are naturally able to support scalability, fault tolerance, and another key properties of cloud-native applications\footnote{\url{https://www.redhat.com/en/topics/cloud-native-apps/stateful-vs-stateless##stateful-vs-stateless}}.

Repour is stateful: it remembers its running (asynchronous) tasks executed in its asyncio event loop. When \textit{POST /cancel/{taskId}} request is being handled, Repour's load-balancer has to send this request to all its underlying replicas since \textbf{at most} one of these replicas is potentially executing the task corresponding to the given \textit{taskId}. On the other hand, Reqour is stateless — hence, it should support clusterability more easily than Repour.

\end{document}
