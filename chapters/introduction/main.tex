\documentclass[../../main.tex]{subfiles}

\begin{document}

In software development, programmers build products by writing source code. This source code is written in specific programming languages and is (often) using technologies/frameworks built upon these programming languages. Probably more often than we would like to, these underlying technologies get outdated and in order to keep the product capable of competition at the market, we are forced to migrate the technologies, sometimes even the programming language itself.

This process is typically not as difficult as the initial creation of the product (since we already have a decent knowledge of what we are going to do, and also hold old source code which can be used during the whole migration process). On the other hand, it is definitely not that trivial to migrate a product, especially in cases when we do change also the programming language itself.

The goal of this thesis is to migrate Repour microservice (which is written in Python) into a new microservice written in Quarkus framework.

The thesis consists of 6 chapters, excluding the conclusion. Chapter \ref{chap:context} brings a reader to the context of the thesis. Chapter \ref{chap:repour} introduces the current implementation. Chapter \ref{chap:reqour} proposes a new implementation. Chapter \ref{chap:comparision} compares both implementations with a focus on maintainability, scalability, and clusterability. Finally, chapter \ref{chap:deployment} describes the deployment of the new implementation.

\end{document}