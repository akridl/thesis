\documentclass[../main.tex]{subfiles}

\begin{document}

Repour is written in Python using asyncio library\footnote{\url{https://docs.python.org/3/library/asyncio.html}}, which is designed to write concurrent code using the \textit{async/await} syntax. The core of asyncio applications is the event loop\footnote{\url{https://docs.python.org/3/library/asyncio-eventloop.html##asyncio-event-loop}}, which takes care of asynchronous tasks, its callbacks, IO operations, and others.

Repour server is being initialized through several methods — below are listed the most important parts of the initialization (all of these listings are taken\footnote{There is not 1:1 correspondence, since pseudocodes covered in this text are simplified.} from Repour's GitHub using \textit{akridl-thesis} tag as the revision: \url{https://github.com/project-ncl/repour/blob/akridl-thesis}).

\begin{lstlisting}[language=Python, caption=Server initialization: \textit{start\_server} method, label={lst:startServer}]
def start_server(bind, repo_provider, repour_url, adjust_provider):
    loop = asyncio.get_event_loop()

    loop.run_until_complete(
        init(
            loop=loop,
            bind=bind,
            repo_provider=repo_provider,
            repour_url=repour_url,
            adjust_provider=adjust_provider,
        )
    )

    try:
        loop.run_forever()
    except KeyboardInterrupt:
        logger.debug("KeyboardInterrupt")
\end{lstlisting}

In the listing \ref{lst:startServer} displaying \textit{start\_server()}\footnote{\url{https://github.com/project-ncl/repour/blob/akridl-thesis/repour/server/server.py##L99}}, we can see:
\begin{itemize}
    \item \textbf{L2:}\\
    Obtaining of asyncio's event loop.

    \item \textbf{L4-12:}\\
    Initialize the application as described at the listing \ref{lst:init}.

    \item \textbf{L14-17:}\\
    Only once was the initialization of the event loop completed, start the server indefinitely. Allow the server to be killed by a keyboard interrupt.
\end{itemize}

\begin{lstlisting}[language=Python, caption=Server intialization: \textit{init} method, label={lst:init}]
async def init(loop, bind, repo_provider, repour_url, adjust_provider):
    app = web.Application(loop=loop)

    external_to_internal_source = endpoint.validated_json_endpoint(
        shutdown_callbacks,
        validation.external_to_internal,
        external_to_internal.translate,
        repour_url,
    )

    clone_source = endpoint.validated_json_endpoint(
        shutdown_callbacks,
        validation.clone,
        clone.clone,
        repour_url,
        send_logs_to_bifrost=False,
    )

    adjust_source = endpoint.validated_json_endpoint(
        shutdown_callbacks, validation.adjust_modeb, adjust.adjust, repour_url
    )

    internal_scm_source = endpoint.validated_json_endpoint(
        shutdown_callbacks,
        validation.internal_scm,
        internal_scm.internal_scm,
        repour_url,
        send_logs_to_bifrost=False,
    )

    app.router.add_route("GET", "/", info.handle_request)
    app.router.add_route(
        "POST", "/git-external-to-internal", external_to_internal_source
    )
    app.router.add_route("POST", "/clone", clone_source)
    app.router.add_route("POST", "/adjust", adjust_source)
    app.router.add_route("POST", "/internal-scm", internal_scm_source)
    app.router.add_route("POST", "/cancel/{task_id}", cancel.handle_cancel)
    app.router.add_route("GET", "/version", info.handle_version)

    server = await loop.create_server(
        app.make_handler(access_log=None), bind["address"], bind["port"]
    )
\end{lstlisting}

In the listing \ref{lst:init} displaying \textit{init()}\footnote{\url{https://github.com/project-ncl/repour/blob/akridl-thesis/repour/server/server.py##L32}}, we can see:
\begin{itemize}
    \item \textbf{L2:}\\
    Initialize \textit{app} as a web server.

    \item \textbf{L4-29:}\\
    Initialize endpoint handlers.

    \item \textbf{L31-40:}\\
    Register previously initialized endpoint handlers as routes.

    \item \textbf{L42-44:}\\
    Finally, create a TCP server bound to the specified address and port.

\end{itemize}

\end{document}
