\documentclass[../main.tex]{subfiles}

\begin{document}

In section \ref{section:pnc}, we described the build process. However, that is not the only process PNC can execute (although being run the most often).

Another such process is the process of repository creation. This is used by productization engineers when they want to create internal SCM repository within PNC workspace, where PNC is allowed to contribute to (e.g. push alignment changes). Such an internal repository is (at least when being created) synchronized to its external repository counterpart.

The repository creation process works (from high-level perspective) as follows:

\begin{enumerate}
    \item \textbf{Orch handles \textit{POST /create-and-sync} request}\\
    This endpoint is triggered either from PNC UI or Bacon CLI

    \item \textbf{Orch triggers repository creation process in BPM}\\
    BPM starts the repository creation process

    \item \textbf{BPM calls \textit{POST /git-external-to-internal}}\\
    This is used for validation whether an internal repository with such an internal URL does not exist already.

    \item \textbf{BPM calls \textit{POST /internal-scm}}\\
    Create new internal SCM repository within PNC workspace.

    \item \textbf{BPM calls \textit{\textit{POST /clone}}}\\
    Clone external repository into internal one.
\end{enumerate}

\begin{figure}
  \begin{center}
    \includegraphics[width=\textwidth]{images/repository-creation-process.png}
  \end{center}
  \caption{Repository creation process}
  \label{fig:repository-creation-process}
\end{figure}

Visual representation of the repository creation process is depicted in Figure \ref{fig:repository-creation-process}.

\end{document}
