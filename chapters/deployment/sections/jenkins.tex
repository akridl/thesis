\documentclass[../main.tex]{subfiles}

\begin{document}

The pipeline is triggered on push to \textit{main} branch. At first, \textit{SNAPSHOT} job is being started. Using maven, it compiles the source code, runs all the tests (both unit and integration ones), and finally, it checks the formatting based on defined template, e.g. \textit{eclipse-codeStyle.xml}\footnote{\url{https://github.com/project-ncl/reqour/blob/akridl-thesis/eclipse-codeStyle.xml}}. In case this succeeds, built artifacts are deployed to JBoss Nexus repository\footnote{\url{https://repository.jboss.org/nexus/content/repositories/snapshots/org/jboss/pnc/reqour/}}. Some of them will be fetched and used in future stages of the pipeline.

If \textit{SNAPSHOT} job is successful, another 3 jobs are run:
\begin{itemize}
    \item \textbf{reqour-image job:} Triggers \textit{reqour-image} pipeline at GitLab using its REST API\footnote{\url{https://docs.gitlab.com/ee/ci/triggers/##trigger-a-pipeline}}.

    \item \textbf{reqour-adjuster-image job}: Same as \textit{reqour-image} job, but triggers \textit{reqour-adjuster-image} pipeline.

    \item \textbf{sonarqube job:} Runs static analysis using Sonarqube.
\end{itemize}

The above description is visualized in Figure \ref{fig:jenkins}.

\begin{figure}
  \begin{center}
    \includegraphics[width=\textwidth]{images/jenkins.png}
  \end{center}
  \caption{Jenkins part of the pipeline}
  \label{fig:jenkins}
\end{figure}

\end{document}
