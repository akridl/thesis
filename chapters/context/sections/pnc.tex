\documentclass[../main.tex]{subfiles}
%% \graphicspath{{\subfix{../images/}}}

\begin{document}

As is clearly described in an IEEE Software article: "Release engineering focuses
on building a pipeline that transforms
source code into an integrated, compiled, packaged, tested, and signed
product that’s ready for release. The
pipeline’s input is the source code developers write to create a product"\cite{releaseEngineeringArticle}. Such release engineers are present in any development team which develops enterprise-ready product. Hence, they are present also in Red Hat Middleware\footnote{\url{https://developers.redhat.com/middleware}} teams. Examples of products from Red Hat Middleware are Quarkus\footnote{\url{https://developers.redhat.com/products/quarkus/overview}} or JBoss EAP\footnote{\url{https://developers.redhat.com/products/eap/overview}}.

In order to make release engineering pipeline effective is to automate as many tasks as possible. Within middleware productization\footnote{This is how is called the Release Engineering discipline in Red Hat} pipeline, one of these automated tasks is done by PNC build system. Hence, "customers" of PNC are middleware productization engineers and (non)functional requirements are created by them. The most notable requirements are:

\begin{itemize}
  \item \textbf{Structured data organization:} Productization engineers can
organize their data into several levels: Products (e.g. JBoss EAP) has multiple Product Versions (e.g. \textit{3.8}). Every product version can have multiple Product Milestones.

  \item \textbf{Ability to build from upstream:} Being able to fetch source code directly from upstream, e.g. GitHub repository, and make the build above any valid revision in there.

  \item \textbf{Re-usability of previously built artifacts:} Previously built artifacts are used by future builds when it is possible.

  \item \textbf{Dependency alignment:} Automatic alignment of public dependencies to Red Hat versions.

  \item \textbf{Build reproducibility:} Ability to re-create previously run build.

  \item \textbf{Support for multiple build tools:} Provide support for Maven, Gradle, NPM and SBT builds.

  \item \textbf{Ability to re-use build configurations:} Build configuration of the concrete product version can be re-used in its product milestones.

\end{itemize}

\subsection*{System architecture}
Microservices

\subsection*{Build process}
foo bar

\end{document}
