\documentclass[../main.tex]{subfiles}

\begin{document}

Jakarta EE specification process\footnote{\url{https://jakarta.ee/committees/specification/guide/}} is long-running and formal, which leads to specifications being mature and stable. Although it may not seem so at first glance, this can be quite limiting disadvantage: typical examples are teams who develop enterprise applications, but want to use latest enterprise standards. Using Jakarta EE, such an integration can take place in a year. This was one of the reasons why Microprofile\footnote{\url{https://microprofile.io/}} got popular. Another one being its specialization on optimizing enterprise Java for a microservices architecture.

From the architectural point of view, Microprofile is built on same principles as Jakarta EE is: consists of specifications and once we use Microprofile API, we have the access to functionality of any its specification.

Quarkus is an open-source implementation\footnote{\url{https://github.com/quarkusio/quarkus}} of Microprofile API and has a great focus on microservices: "Quarkus is a Java framework that targets microservices and serverless system
development. Quarkus emerged as an alternative to the existing Java
microservices stacks to provide an application framework that delivers an
unmatched performance benefits while still providing a development model
utilizing the APIs (Application-Programming interfaces) of popular libraries
and Java standards that the Java ecosystem has been practicing for years."\cite{quarkusinaction}\cite{quarkusBook}.

Documentation is separated as a series of guides\footnote{\url{https://quarkus.io/guides/}}. When describing implementation details later in the text, we will often reference concrete guides from this series.

\end{document}
