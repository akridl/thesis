\documentclass[../main.tex]{subfiles}

\begin{document}

As was already briefly mentioned in \ref{subsubsec:during-debugging}, version endpoint serves for debugging purposes, e.g. in case a PNC developer wants to check that the deployment took place and the newest version of the application has been deployed. The version endpoint is not Reqour-specific but is present in PNC microservices thoroughly.

The endpoint responds with JSON of the following format:

\begin{lstlisting}[numbers=none]
{
  "name": "string",
  "version": "string",
  "commit": "string",
  "builtOn": "2022-03-10T12:15:50-04:00",
  "components": [
    "string"
  ]
}
\end{lstlisting}

where the fields have the following meaning:
\begin{itemize}
    \item \textbf{name:} Name of the application.
    
    \item \textbf{version:} Version of the application.

    \item \textbf{commit:} SHA of the commit from which the deployment comes from.

    \item \textbf{builtOn:} Time of the build in \textit{ISO8601}\footnote{\url{https://datatracker.ietf.org/doc/html/rfc3339}} format.

    \item \textbf{components:} List of application components in the same format as this DTO.
\end{itemize}

For example, the current version of the deployment at devel environment is:
\begin{lstlisting}[numbers=none]
{
  "name": "reqour-rest",
  "version": "1.0.0-SNAPSHOT",
  "commit": "32739fbde8fd403dec912c96ea917690189ccb5e",
  "builtOn": "2024-11-28T10:48:02Z",
  "components": []
}
\end{lstlisting}

From the above output, we can easily deduce that the current version comes from the commit \url{https://github.com/project-ncl/reqour/commit/32739fbde8fd403dec912c96ea917690189ccb5e}.

\end{document}
