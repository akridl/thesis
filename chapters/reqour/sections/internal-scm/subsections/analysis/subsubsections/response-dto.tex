\documentclass[../main.tex]{subfiles}

\begin{document}

The response DTO (sent as callback payload) is JSON of the following format:
\begin{lstlisting}[numbers=none]
{
    "readonlyUrl": "string",
    "readwriteUrl": "string",
    "status": {
        "enum": [
            "SUCCESS_CREATED",
            "SUCCESS_ALREADY_EXISTS",
            "FAILED"
        ],
        "type": "string"
    },
    "callback": {
        "status": {
            "enum": [
                "SUCCESS",
                "FAILED",
                "TIMED_OUT",
                "CANCELLED",
                "SYSTEM_ERROR"
            ],
            "type": "string"
        },
        "id": "string"
    }
}
\end{lstlisting}

Fields have the following meaning:
\begin{itemize}
    \item \textbf{readonlyUrl}\\
    Read-only URL of the repository, used e.g. for cloning.

    \item \textbf{readwriteUrl}\\
    Read-write URL of the repository, used e.g. for pushing.

    \item \textbf{status}:\\
    Status of the creation operation. This field is used to differentiate whether the requested repository is created or already existed.

    \item \textbf{callback}:\\
    Callback describing resulting status of the Reqour's async task.
\end{itemize}

\end{document}
