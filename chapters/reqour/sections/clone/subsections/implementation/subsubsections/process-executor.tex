\documentclass[../main.tex]{subfiles}

\begin{document}

The implementation of the process executor is shown in the Listing \ref{lst:process-executor}\footnote{\url{https://github.com/project-ncl/reqour/blob/akridl-thesis/core/src/main/java/org/jboss/pnc/reqour/common/executor/process/ProcessExecutorImpl.java##L35-L73}}. It uses \textit{ProcessBuilder}\footnote{\url{https://docs.oracle.com/en/java/javase/21/docs/api/java.base/java/lang/ProcessBuilder.html}} to create instances of \textit{Process}\footnote{\url{https://docs.oracle.com/en/java/javase/21/docs/api/java.base/java/lang/Process.html}} class. Furthermore, similarly as asynchronous task executor (see \ref{lst:task-executor}), it uses completable futures and container's managed executor in order to create asynchronous tasks. This is hidden under the hood of calls \textit{createOutputConsumerProcess} at lines 7 and 10.

\begin{lstlisting}[language=Java, caption=Process executor implementation, label={lst:process-executor}]
public int execute(ProcessContext processContext) {
    ProcessBuilder processBuilder = new ProcessBuilder(processContext.getCommand())
            .directory(processContext.getWorkingDirectory().toFile());
    processBuilder.environment().putAll(processContext.getExtraEnvVariables());

    Process processStart = processBuilder.start();
    CompletableFuture<Void> stdoutConsumerProcess = createOutputConsumerProcess(
            processStart.inputReader(),
            processContext.getStdoutConsumer());
    CompletableFuture<Void> stderrConsumerProcess = createOutputConsumerProcess(
            processStart.errorReader(),
            processContext.getStderrConsumer());

    int exitCode = processStart.waitFor();
    stdoutConsumerProcess.join();
    stderrConsumerProcess.join();

    return exitCode;
}
\end{lstlisting}

\end{document}
