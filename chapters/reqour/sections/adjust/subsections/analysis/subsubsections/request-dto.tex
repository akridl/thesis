\documentclass[../main.tex]{subfiles}

\begin{document}

Adjust request DTO is JSON of the following format:

\begin{lstlisting}[numbers=none]
{
    "originRepoUrl": "string",
    "ref": "string",
    "sync": boolean,
    "buildType": "string",
    "tempBuild": boolean,
    "internalUrl": {
        "readwriteUrl": "string",
        "readonlyUrl": "string"
    },
    "brewPullActive": boolean,
    "buildConfigParameters": {
	"string",
        "string"
    },
    "pncDefaultAlignmentParameters": "string",
    "taskId": "string",
    "callback": {
        "method": "string",
        "uri": "string"
    }
}
\end{lstlisting}

where DTO fields have the following meaning:
\begin{itemize}
    \item \textbf{originRepoUrl}\\
    External repository URL from which to synchronize to the corresponding internal repository.

    \item \textbf{sync}\\
    Whether to synchronize the internal (downstream) repository from the external (upstream) repository.

    \item \textbf{buildType}\\
    Build type, e.g. Maven.

    \item \textbf{tempBuild}\\
    Whether the build is temporary. Hence, whether the alignment should work with persistent or temporary Red Hat versions.

    \item \textbf{internalUrl}\\
    Both read-only and read-write URLs of the internal repository.

    \item \textbf{brewPullActive}\\
    Whether DA should look for built artifacts in Brew (Brew is just another build system among PNC).

    \item \textbf{buildConfigParameters}\\
    Parameters specified within the build configuration when triggering the build. Alignment might take some of these for configuration.

    \item \textbf{pncDefaultAlignmentParameters}\\
    Sane defaults set by PNC, which are not possible to override.

    \item \textbf{ref, taskId, callback}\\
    Same meaning as in Cloning DTO described in \ref{subsubsec:cloning-request-dto}.
\end{itemize}

\end{document}
